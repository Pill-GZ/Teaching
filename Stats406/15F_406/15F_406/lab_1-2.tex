\documentclass[10pt]{article}
\setlength{\oddsidemargin}{0.0in} \setlength{\evensidemargin}{0.0in}
\setlength{\textwidth}{6.5in} \setlength{\topmargin}{-0.25in}
\setlength{\textheight}{8.75in}
\usepackage{amsmath, amssymb, amsfonts, amscd, xspace, pifont, natbib, fullpage}
\usepackage{epsfig, amsfonts, verbatim, multirow}
\usepackage{epstopdf}
\usepackage{listings}
\lstset{
  basicstyle=\ttfamily,
  mathescape
}
%\usepackage{setspace}
\newcommand{\mycite}[1]{{\citeNP{#1}}}

\parindent=0pt

%%%%%%%%%%%%%%%%%%%%%%%%%%%%%%%%%%%%%%%%%%%%%%%%%%%%%%%%%%%%%%%%%%%%%%%%%%%%%%%%%%%%%%%%%%%%%%%%%%%
% Different font in captions
\newcommand{\captionfonts}{\small}
\makeatletter  % Allow the use of @ in command names
\long\def\@makecaption#1#2{%
  \vskip\abovecaptionskip
  \sbox\@tempboxa{{\captionfonts #1: #2}}%
  \ifdim \wd\@tempboxa >\hsize
    {\captionfonts #1: #2\par}
  \else
    \hbox to\hsize{\hfil\box\@tempboxa\hfil}%
  \fi
  \vskip\belowcaptionskip}
\makeatother   % Cancel the effect of \makeatletter
%%%%%%%%%%%%%%%%%%%%%%%%%%%%%%%%%%%%%%%%%%%%%%%%%%%%%%%%%%%%%%%%%%%%%%%%%%%%%%%%%%%%%%%%%%%%%%%%%%%

%% Matrix, Vector
\newcommand{\V}[1]{\ensuremath{\boldsymbol{#1}}\xspace}
\newcommand{\M}[1]{\ensuremath{\boldsymbol{#1}}\xspace}
%% Math Functions
\newcommand{\F}[1]{\ensuremath{\mathrm{#1}}\xspace}
\newcommand{\sgn}{\F{sgn}}
\newcommand{\tr}{\F{trace}}
\newcommand{\diag}{\F{diag}}
\newcommand{\dett}{\F{det}}
%% Transpose
\newcommand{\T}[1]{\ensuremath{{#1}^{\mbox{\sf\tiny T}}}}

%%
\def  \R  {\boldsymbol R}
\def\bX{\boldsymbol X}
\def\bY{\boldsymbol Y}
\def\bbeta{\boldsymbol \beta}
\def\blambda{\boldsymbol \lambda}
\def\bepsilon{\boldsymbol \epsilon}
\def\bone{\boldsymbol{1}}
\def\bzero{\boldsymbol 0}
\def\E{\mbox{E}}
\def\var{\mbox{var}}
\def\gauss{\mbox{N}}
\def\lap{\mbox{L}}
\def\G{\mbox{G}}
\def\go{  $\R$  ightarrow}
\def\invG{\mbox{G}^{-1}}
\def\argmin{\arg\min}

\newtheorem{theorem}{Theorem}
\newtheorem{proposition}{Proposition}
\newtheorem{lemma}{Lemma}
\newtheorem{corollary}{Corollary}
\newtheorem{algorithm}{Algorithm}
\newtheorem{definition}{Definition}
\newtheorem{remark}{Remark}

\usepackage{color}
\definecolor{codegreen}{rgb}{0,0.6,0}
\definecolor{codegray}{rgb}{0.5,0.5,0.5}
\definecolor{codepurple}{rgb}{0.58,0,0.82}
\definecolor{backcolour}{rgb}{0.95,0.95,0.92}
\lstdefinestyle{displaycode}{
    backgroundcolor=\color{backcolour},   
    commentstyle=\color{codegreen},
    keywordstyle=\color{magenta},
    numberstyle=\tiny\color{codegray},
    stringstyle=\color{codepurple},
    basicstyle=\footnotesize,
    breakatwhitespace=false,         
    breaklines=true,                 
    captionpos=b,                    
    keepspaces=true,                 
%    columns=flexible,
    numbers=none,
    numbersep=5pt,                  
    showspaces=false,                
    showstringspaces=true,
    showtabs=false,                  
    tabsize=2%,
    %xleftmargin=2em,
    %xrightmargin=2em,
} 

\lstset{style=displaycode}
%\newcommand{\displaycodefile}[1]{\lstinputlisting[language=R]{./codeblocks/#1.txt}}

\begin{document}

\title{\Large \bf STATS 406 F15: Lab 01}
\date{}

\maketitle
%\setcounter{equation}{0}
\section{Section inforamtion}
\begin{itemize}
	
	\item Lab section: Thursday, 8:30-10:00 am  (Yuan Zhang)
	\item Office hours: (Science Learning Center, Chemistry building, attend any)
	\begin{itemize}
	\item Tuesday: 10:00a.m. -- 11:30a.m. (Yuan Zhang)
	\item Tuesday: 1:00p.m. -- 2:30p.m. (Dao Nguyen)
	\item Tuesday: 3:00p.m. -- 4:30p.m.  (Jun Guo)
	\end{itemize}
 \end{itemize}
\section{Start coding in Rstudio}
\begin{itemize}
	\item New code:
	\begin{itemize}
		\item By shortcut: (Ctrl + Shift + N).
		\item By menu: File $\to$ New File $\to$ R Script.
	\end{itemize}
	\item Save the code:
	\begin{itemize}
		\item By shortcut: (Ctrl + S).
		\item By menu: File $\to$ Save.
	\end{itemize}
	\item Execute the code:
	\begin{itemize}
		\item Line-by-line:
		\begin{enumerate}
			\item Move the cursor to the line.
			\item Press (Ctrl + R) for Windows or (Command + Enter) for Mac.
		\end{enumerate}
		\item Run script file(Recommended):
		\begin{enumerate}
			\item Set proper working directory.
\begin{lstlisting}[style=displaycode, language=R]
	# Only for illustration, will not run
	## Check the working directory
	> getwd();
	[1] "C:/Users/yzhanghf/Documents"
	## Set proper working directory
	> setwd('C:/Users/yzhanghf/Desktop');
\end{lstlisting}
			\item Source the file.
%			\displaycodefile{1_02}
\begin{lstlisting}[style=displaycode, language=R]
	> source('Lab_1.r');
\end{lstlisting}
		\end{enumerate}
	\end{itemize}
\end{itemize}

\section{Variables and Functions}
\subsection{Variables}
Two ways to assign a value to a variable
\begin{itemize}
	\item (variable) \texttt{=} (value)
	\item (variable) \texttt{<-} (value)
\end{itemize}

\begin{lstlisting}[style=displaycode, language=R]
	> x = 6;
	> y = x;
	> print(y);
	[1] 6
	
	> x <- 8;
	> y <- x^2 - 2*x + 1;
	> print(y);
	[1] 49
\end{lstlisting}

Do not use reserved keywords of R as variable names. Examples:
\begin{itemize}
	\item c, function, print, return, $\ldots$
	\item sin, cos, sqrt, abs, $\ldots$
	\item plot, maplot, levelplot, hist, $\ldots$
\end{itemize}

{\bf Quiz: guess the output before running the script:}
\begin{lstlisting}[style=displaycode, language=R]
	# What happens if you run this?
	> c = 3;
	> aa = c(1, 2, c);
	> print(aa);
\end{lstlisting}
Run:
\begin{lstlisting}[style=displaycode, language=R]
	> rm(c);
\end{lstlisting}
afterwards to remove the user-specified definition for ``c''.

\subsection{Define a user-specified function}
\underline{Format:} (function name) = function(argument list)\{function content; return(variable);\}
\begin{itemize}
	\item Simple functions can be defined in the in the main code file, while complicated functions are advised to be defined in separate files for clarity.
\end{itemize}
\underline{Example}: create a file ``plusthree.r'' under working directory with the following content:
\begin{lstlisting}[style=displaycode, language=R]
	# Content of file: plusthree.r
	plusthree = function(x, y=3){
		z = x + y;
		return(z);
	}
\end{lstlisting}
Then in the console:
\begin{lstlisting}[style=displaycode, language=R]
	> source('plusthree.r');
	> plusthree(3);
	[1] 6
	> plusthree(3, 6);
	[1] 9
\end{lstlisting}
Remarks:
\begin{enumerate}
	\item You can also run the content of plusthree.r instead of sourcing it in the console.
	\item Assign \emph{default values} to variables while defining the argument list.
	\item In STATS 406, R does not pass variables by reference, so always have {\bf return} by the end.
\end{enumerate}

\section{Vectors and Matrices}
\subsection{Create a vector}
Basic ways to create vectors:
\begin{itemize}
	\item {\bf c()}, example {\bf a=c(1, 2)}.
	\item {\bf rep()}, example {\bf a=rep(1, 3)}, creates an all-one vector of length $3$.
	\item {\bf seq()}, example {\bf a=seq(from=0, to=2, by=0.1)} or {\bf a=seq(0, 2, 0.1)}(not recommended).
	The reason that we advise writting ``from'', ``to'' and especially ``by'' in {\bf seq} is \underline{\bf clarity}.
	{\bf Quiz: what does this give you, ``0,3,6'' or ``0,2,4,6''?}
	\begin{lstlisting}[style=displaycode, language=R]
	# Not recommended way of writing
	> print(seq(0, 6, 3));
	\end{lstlisting}
	An alternative to ``by'' is ``length.out''.
	\begin{lstlisting}[style=displaycode, language=R]
	> print(seq(from=0, to=6, length.out=7));
	[1] 0 1 2 3 4 5 6
	\end{lstlisting}
	\item Integer sequences, example {\bf a=1:4;} equivalent to {\bf a=c(1, 2, 3, 4);} This method is originally designed to facilitate {\bf for} loops, but is now more widely applied.
	\item By random number generators. Examples: {\bf a=rnorm(1000)}, {\bf a=runif(100, -0.5, 0.5)} and so on. A few important notes:
	\begin{itemize}
		\item {\bf Always save random seed if any part of your code involves random number generation to allow others repeat your experiment and verify your results!}
\begin{lstlisting}[style=displaycode, language=R]
	> set.seed(2015); # Use your lucky number in exams, if you have one.
\end{lstlisting}
		\item {\bf runif(100)} and {\bf rnorm(100)} are OK. {\bf runif(100, -1, 1)} is OK. But {\bf rnorm(100, 1, 3)} should be replaced by {\bf rnorm(100, mean=1, sd=3)}, especially ``sd='' should never be omitted in general cases.
		\begin{itemize}
			\item For all other distributions, explicitly write all parameter names.
		\end{itemize}
	\end{itemize}
\end{itemize}


%\begin{center}
%\framebox[0.75\textwidth][c]{Correctness and clarity are top priorities. Everything else is secondary.}
%\end{center}



\subsection{Vector operations and manupulations}
In R, ordinary operations on vectors are \underline{\bf element-wise}.
\begin{lstlisting}[style=displaycode, language=R]
	> a1 = c(1, 2, 3); a2 = c(1, 0, -1);
	> print(a1*a2);
	[1] 1 0 -3
\end{lstlisting}
Operations between vectors of different lengths is $\ldots$
\begin{enumerate}
	\item allowed, if one's length is multiple to the other's
	\begin{lstlisting}[style=displaycode, language=R]
		> a1 = c(1, 2, 3); a2 = c(1, 0, -1);
		> a3 = c(-1, 0, 1, a2);
		# Element-wise product of vectors with different lengths
		> print(a1*a3);
		[1] -1  0  3  1  0 -3
	\end{lstlisting}
	\item but should be avoided, to reduce bugs that do not trigger error messages.
	\item The only exception is an operation between a vector and a scalar.
	\begin{lstlisting}[style=displaycode, language=R]
		> a1 = c(1, 2, 3);
		> print(a1+3);
		[1] 4 5 6
	\end{lstlisting}
	Such operations are fully acceptable and widely-used. In R, scalars are treated as length-one vectors:
\begin{lstlisting}[style=displaycode, language=R]
	> a=1;
	> is.vector(a);
	[1] TRUE
	> length(a);
	[1] 1
\end{lstlisting}
\end{enumerate}

Notice: an operation alone does not change the vector itself.
\begin{lstlisting}[style=displaycode, language=R]
	> a1 = c(1, 2, 3);
	> append(a1, 3);
	[1] 1 2 3 3
	> print(a1);
	[1] 1 2 3
	
	# If you want to permanently append 3 to a1, then use ``=''.
	> a1 = append(a1, 3);
	> print(a1);
	[1] 1 2 3 3
\end{lstlisting}

Two basic ways to extract subvectors:
\begin{enumerate}
	\item By integer indexing: {\bf a1[2:3]}, {\bf a1[c(1, 3)]} and so on.
	\item By logical indexing: {\bf a1[c(TRUE, FALSE, FALSE)]}. Accessing by logical expressions are essentially the same.
\begin{lstlisting}[style=displaycode, language=R]
	> a1 = c(1, 2, 3); a2 = c(1, 0, -1);
	> print( a1[a2<0.5] );
	[1] 2 3
	
	> indexvector = a2>0.5;
	> print(indexvector);
	[1] TRUE FALSE FALSE
	> print( a1[indexvector] );
	[1] 1
\end{lstlisting}

{\bf Quiz(seen in lecture): what will be the output?}
\begin{lstlisting}[style=displaycode, language=R]
	> a1=c(1, 2, 3);
	> indexvector=c(TRUE, FALSE); # Incorrectly short in length.
	# Only for illustration, avoid in practical programming
	> print( a1[indexvector] );
\end{lstlisting}
This is an example of a potential mistake that does not trigger error message.

\end{enumerate}

Many functions defined for scalars accept vectors as input:
\begin{lstlisting}[style=displaycode, language=R]
	> a=seq(from=0, to=pi, by=0.25*pi);
	> print(round(sin(a), 3));
	[1] 0.000 0.707 1.000 0.707 0.000
\end{lstlisting}
The function {\bf sin} was applied element-wise.

Many other useful operations, including {\bf sort()}, {\bf order()}, {\bf min()}, {\bf median()} and so on. Learn their usage from their manuals.
\begin{lstlisting}[style=displaycode, language=R]
	# Toggle documentation, press ``q'' to quit when finished.
	> ?sort
\end{lstlisting}

\subsection{Create a matrix}

Useful commands:
\begin{itemize}
	\item {\bf matrix()}, first specify \#rows, then \#columns. Example:
\begin{lstlisting}[style=displaycode, language=R]
	> a = matrix(1:6, c(2, 3));
	> print(a);
	           [,1] [,2] [,3]
	     [1,]    1    3    5
	     [2,]    2    4    6
\end{lstlisting}
	\item {\bf array()}, similar to {\bf matrix}, but can easily generate matrices of identical elements:
\begin{lstlisting}[style=displaycode, language=R]
> a = array(0, c(2,2));
> print(a);
           [,1] [,2]
     [1,]    0    0
     [2,]    0    0
\end{lstlisting}
	{\bf Quiz: how to generate an all-one matrix of $6\times 6$ dimensions with ``matrix()''?}
	\item {\bf cbind()} and {\bf rbind()}, combine vectors and/or matrices of proper dimensions. Example:
\begin{lstlisting}[style=displaycode, language=R]
	> a1 = 1:3; a2 = 2:4;
	> a12 = cbind(a1, a2);
	> print(a12);
	          a1 a2
	     [1,]  1  2
	     [2,]  2  3
	     [3,]  3  4
	> print(cbind(a12, a1));
	          a1 a2 a1
	     [1,]  1  2  1
	     [2,]  2  3  2
	     [3,]  3  4  3
\end{lstlisting}
\end{itemize}

\subsection{Matrix operations and manipulations}
Like vectors, operations between scalars applied to matrices become element-wise.

The symbol for matrix product is {\bf \%*\%}, example: {\bf A\%*\%B}.

Two important remarks:
\begin{enumerate}
	\item Non-degenerate submatrix views produce matrix objects, but degenerate submatrix views produce vectors. To prevent this, use {\bf drop=FALSE}.
\begin{lstlisting}[style=displaycode, language=R]
	> A = array(1:9, c(3,3));
	> is.matrix(A[1:2, 1:2]);
	[1] TRUE
	> is.matrix(A[, 1]);
	[1] FALSE
	> is.matrix(A[, 1, drop=FALSE]);
	[1] TRUE
\end{lstlisting}
	\item R treat all vectors as column vectors, but it does not differentiate between row- and column- vectors.
\begin{lstlisting}[style=displaycode, language=R]
	# Example 1
	> a1 = 1:3;
	> a2 = t(a1); # transpose
	> print(a2);
	          [,1] [,2] [,3]
	     [1,]    1    2    3
	> is.vector(a2); # After transpose, a2 becomes a matrix.
	[1] FALSE
	
	# Example 2
	> A = diag(1:3);
	> print( a1 %*% A %*% a1 );
	          [,1]
	     [1,]   36
\end{lstlisting}
	No need to transpose the first ``a1'' in the sandwich product.
\end{enumerate}

\subsection{Data frames}
Data frame is an extension to matrices, allowing columns to take different types.

\begin{lstlisting}[style=displaycode, language=R]
	> A1 = (1:5)/10; A2 = letters[1:5]; A3 = round(sin(1:5), 3);
	# Numerical objects are converted to characters when concatenating A1 to A3.
	> A = cbind(A1, A2, A3);
	> print(A);
	          A1    A2  A3      
	     [1,] "0.1" "a" "0.841" 
	     [2,] "0.2" "b" "0.909" 
	     [3,] "0.3" "c" "0.141" 
	     [4,] "0.4" "d" "-0.757"
	     [5,] "0.5" "e" "-0.959"
	# Data frames keep numerical variables numerical, and convert characters to factors.
	> Adataframe = data.frame(A1, A2, A3);
	> print(Adataframe);
	      A1 A2     A3
	   1 0.1  a  0.841
	   2 0.2  b  0.909
	   3 0.3  c  0.141
	   4 0.4  d -0.757
	   5 0.5  e -0.959
\end{lstlisting}

Variables in a data frame can be conveniently referred by their names, like {\bf A\$A2}.

Setting input type to data.frame is a requirement in many packages for regression and other purposes.

\end{document}
