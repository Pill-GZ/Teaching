\documentclass[12pt]{article}
\setlength{\oddsidemargin}{0.0in} \setlength{\evensidemargin}{0.0in}
\setlength{\textwidth}{6.5in} \setlength{\topmargin}{-0.25in}
\setlength{\textheight}{8.75in}
\usepackage{amsmath, amssymb, amsfonts, amscd, xspace, pifont, natbib}
\usepackage{epsfig, amsfonts, verbatim, multirow}
\usepackage{epstopdf}
%\usepackage{setspace}
\newcommand{\mycite}[1]{{\citeNP{#1}}}

%%%%%%%%%%%%%%%%%%%%%%%%%%%%%%%%%%%%%%%%%%%%%%%%%%%%%%%%%%%%%%%%%%%%%%%%%%%%%%%%%%%%%%%%%%%%%%%%%%%
% Different font in captions
\newcommand{\captionfonts}{\small}
\makeatletter  % Allow the use of @ in command names
\long\def\@makecaption#1#2{%
  \vskip\abovecaptionskip
  \sbox\@tempboxa{{\captionfonts #1: #2}}%
  \ifdim \wd\@tempboxa >\hsize
    {\captionfonts #1: #2\par}
  \else
    \hbox to\hsize{\hfil\box\@tempboxa\hfil}%
  \fi
  \vskip\belowcaptionskip}
\makeatother   % Cancel the effect of \makeatletter
%%%%%%%%%%%%%%%%%%%%%%%%%%%%%%%%%%%%%%%%%%%%%%%%%%%%%%%%%%%%%%%%%%%%%%%%%%%%%%%%%%%%%%%%%%%%%%%%%%%

%% Matrix, Vector
\newcommand{\V}[1]{\ensuremath{\boldsymbol{#1}}\xspace}
\newcommand{\M}[1]{\ensuremath{\boldsymbol{#1}}\xspace}
%% Math Functions
\newcommand{\F}[1]{\ensuremath{\mathrm{#1}}\xspace}
\newcommand{\sgn}{\F{sgn}}
\newcommand{\tr}{\F{trace}}
\newcommand{\diag}{\F{diag}}
\newcommand{\dett}{\F{det}}
%% Transpose
\newcommand{\T}[1]{\ensuremath{{#1}^{\mbox{\sf\tiny T}}}}

%%
\def\bX{\boldsymbol X}
\def\bY{\boldsymbol Y}
\def\bbeta{\boldsymbol \beta}
\def\blambda{\boldsymbol \lambda}
\def\bepsilon{\boldsymbol \epsilon}
\def\bone{\boldsymbol{1}}
\def\bzero{\boldsymbol 0}
\def\E{\mbox{E}}
\def\var{\mbox{var}}
\def\gauss{\mbox{N}}
\def\lap{\mbox{L}}
\def\G{\mbox{G}}
\def\go{\rightarrow}
\def\invG{\mbox{G}^{-1}}
\def\argmin{\arg\min}


\newtheorem{theorem}{Theorem}
\newtheorem{proposition}{Proposition}
\newtheorem{lemma}{Lemma}
\newtheorem{corollary}{Corollary}
\newtheorem{algorithm}{Algorithm}
\newtheorem{definition}{Definition}
\newtheorem{remark}{Remark}


\begin{document}

%\baselineskip = 1.\baselineskip
%\baselineskip = 2.0\baselineskip
%\doublespacing

%\begin{titlepage}
\title{\Large \bf STATS 406W14 Lab 11}
\date{}

\maketitle
%\bigskip
%\bigskip

%\newpage
%\par\noindent
%{\sc Summary. \\

%\bigskip
%\medskip
%%%%%%%%%%%%%%%%%%%%%%%%%%%%%%%%%%%%%%%%%%%%%%%%%%%%%%%%%%%555555555555555555555555555555555555555555555555555555555555555555555555555555555555555
%\par\noindent
%GLasso; Precision matrix; Spectral clustering; Commute similarity.

%\thispagestyle{empty}
%\end{titlepage}

\section{Recap: SQL }

\begin{itemize}
	\item RSQLite package helps with the database
handling functionality.

\begin{verbatim}
install.packages("RSQLite", dependencies=T)
\end{verbatim}

\item To use the installed RSQLite package you would type
\begin{verbatim}
library(RSQLite)
\end{verbatim}

\item Reading in and initializing databases in R:
\begin{itemize}
	\item Specify that we are using the SQLite database handler to connect to the database, say baseball.db:
\begin{verbatim}
# the database driver is called SQLite
drive <- dbDriver("SQLite")

# establish a connection with the baseball database
connect <- dbConnect(drive, "baseball.db")
\end{verbatim}

\item To see the various tables:
\begin{verbatim}
dbListTables(connect)
\end{verbatim}

\item to see what variables are stored
in the "Salaries" table you would type
\begin{verbatim}
dbListFields(connect, "Salaries")
[1] "yearID" "teamID" "lgID" "playerID" "salary"
\end{verbatim}
\item To obtain one entire table: 
\begin{verbatim}
Salaries = dbReadTable(connect, "Salaries")
\end{verbatim}
\end{itemize}
\item Querying databases in R:
\begin{itemize}
	\item A query, or search, has the basic syntax:
\begin{verbatim}
dbGetQuery(dbconnection, SQLline)
\end{verbatim}
\item The basic psuedocode for the second argument of dbGetQuery is
\begin{verbatim}
"SELECT column(s) from table(s) WHERE constraint(s)"

e.g.,

# Extract player IDs from table Allstar for AL players after 1972
D.AL1972 <- dbGetQuery(connect,
"SELECT playerID from Allstar WHERE lgID in (�AL�) AND yearID>1972")

\end{verbatim}
\end{itemize}
\end{itemize}

\noindent
A few more SQL options:
\begin{itemize}
	\item Ordering the results (ORDER BY):
\begin{verbatim}
SELECT  name, stock price
FROM    Company
WHERE   country=�USA� AND stockPrice > 50
ORDERBY country, name
\end{verbatim}
Ordering is ascending, unless you specify the DESC keyword.
Ties are broken by the second attribute on the ORDERBY list, etc.
\item Aggregation (SUM)
\begin{verbatim}
SELECT  Sum(price)
FROM      Product
WHERE   manufacturer=�Toyota�
\end{verbatim}
SQL supports several aggregation operations:

SUM, MIN, MAX, AVG, COUNT


\item Extracting data from two tables (Joins):
e.g., Two tables
\begin{verbatim}
Product (name,  price, category, maker)
Purchase (buyer,  seller,  store,  product)
\end{verbatim}

\begin{verbatim}
SELECT   name, store
FROM      Person, Purchase
WHERE    name=buyer AND city=�Seattle�
                  AND product=�gizmo�

\end{verbatim}


\item Grouping and Aggregation (GROUP BY)
\begin{verbatim}
SELECT       product, Sum(price)
FROM          Product,  Purchase
WHERE       Product.name = Purchase.product
GROUPBY  Product.name
\end{verbatim}
The above finds out how much was sold of every product.

\item Subqueries:
\begin{verbatim}
SELECT  Company.name
 FROM   Company, Product
 WHERE  Company.name=maker
        AND  Product.name  IN
                          (SELECT product
                           FROM   Purchase
                           WHERE  buyer = �John Doe�);
\end{verbatim}


\end{itemize}

\noindent
Example: Open database baseball.db. Using R interface for SQLite database to do the following manipulations:
\begin{enumerate}
  \item Check which tables are involved.
  \item From table ``Master'', for each college, compute the average weight and height of the players born after 1980.
  \item From table ``Master'', select field birthyear for those rows where players come from ``Michigan'' or ``Michigan State''.
  \item From table ``Master'', compute the number of players born in each year.      
\end{enumerate}

\begin{verbatim}

library('RSQLite')

## Set up the connection
drv=dbDriver("SQLite")
conn=dbConnect(drv,"baseball.db")

## Like at what tables we have
dbListTables(conn)

## For each table, look at what fields there are
dbListFields(conn,"Master")
dbListFields(conn,"Teams")

## From table Master, compute the average weight of the players from
## each college and were born after 1980
dt1 = dbGetQuery(conn, "SELECT college, AVG(weight), AVG(height) from Master
where birthYear>1980 GROUP BY college")

## From table Master, select field birthyear for those rows
## where players come from Michigan or Michigan State
dt2 = dbGetQuery(conn,"SELECT birthYear from Master where
college in ('Michigan','Michigan State')")

## From table Master, compute the number of players born in each year
dt3 = dbGetQuery(conn, "SELECT birthYear, COUNT(playerID)
from Master GROUP BY birthYear")


## Close the connection
dbDisconnect(conn)
dbUnloadDriver(drv)
\end{verbatim}


\section{ A few things about XML}
\begin{itemize}
	\item XML is widely used for exchanging information on the World Wide
Web. It stands for eXtensible Markup Language. 
\item The user can define his own structure according to his purpose.

\item Markup Language is a system of how the document is to be described or
logically structured, e.g., HTML and XML.
HTML is used for displaying data on the web browser,
XML is used for carrying data on the web.

\item Structure of XML
\begin{itemize}
	\item Declaration: XML document declaration includes the information of the XML
version (and can also include info on how it is encoded, document type definition etc.)
\begin{verbatim}
<?xml version=�1.0�?>
\end{verbatim}
Any instructions
declared between $<?$ and $?>$ is called as processing instruction.
	\item Comments: Comments are used to inform the user of XML data or help the grader understand your code.
	All XML comments begin with $<!--$ and close with $-->$.

\begin{verbatim}
<!-- This is a comment -->
\end{verbatim}
	
	\item Elements and Attributes: Elements contain the actual data of XML document. Elements
must have a start-tag and an end-tag which contain the element�s
name. The content sits between these two tags. Elements have a
tree-based data structure.

\begin{verbatim}
<?xml version = �1.0� ?>
<BARS> <!-- This is a root tag -->
  <BAR id = "412"><NAME>Joe�s Bar</NAME>
    <BEER><NAME>Bud</NAME>
     <PRICE>2.50</PRICE></BEER>
    <BEER><NAME>Miller</NAME>
      <PRICE>3.00</PRICE></BEER>
  </BAR>
	
	<BAR>...
	...
	... 
</BARS> <!-- This is the end of the  root tag -->
\end{verbatim}
In the example above, BARS is the root, BAR are the children/elements, BEERS are sub-elements etc. 
There is one root per XML document. 

In the start-tag of the BAR element, there is one piece of
information, id. The information in the start-tag is called Attribute. The value of attribute must be quoted, using either single
quotes or double quotes. 

\noindent
\begin{verbatim}
<element_name attribute_name = "attribute value"> Content </element_name>
\end{verbatim}

\item Entities: Entities are often used in XML to represent single special characters which either are not easily typed or have a preassingned meaning in the language (e.g. $>$ and $<$). Entity reference consists of the type $"\& entity_name; "$Some special characters� entity values are predefined. For
example, the ampersand is defined as $"\& amp ;"$. Undefined special
characters can also be defined (Later).


\end{itemize}
\item Parse XML data: A parser is a component  of the complier which checks for correct syntax and builds a data structure. The R package XML provides the required parser and extends R�s capability in processing
XML files.

\begin{verbatim}
## Load the library and parse the XML file
library(XML)
doc <- xmlTreeParse('people.xml')
\end{verbatim}

\item Extracting the root:
\begin{verbatim}
root <- xmlRoot(doc)
\end{verbatim}

\item Extract all the children of the first child of the root:
\begin{verbatim}
ch <- xmlChildren(root[[1]])
\end{verbatim}


\item Get the name of the first node:
\begin{verbatim}
xmlName(root[[1]])
[1] "person"
\end{verbatim}

\item Extract all the children of the first child of the root:
\begin{verbatim}

ch <- xmlChildren(root[[1]])
ch
$role
<role type="rep" startdate="2009-01-06" enddate="2010-03-01" party="Democrat" 
state="HI" district="1" url="http://www.house.gov/abercrombie"/>

attr(,"class")
[1] "XMLNodeList"

\end{verbatim}

\item Get the attributes 
\begin{verbatim}

 xmlAttrs(root[[1]])
                               id                          lastname 
                         "400001"                     "Abercrombie" 
                        firstname                          birthday 
                           "Neil"                      "1938-06-26" 
                           gender                             pvsid 
                              "M"                           "26827" 
                             osid                        bioguideid 
                      "N00007665"                         "A000014" 
                        metavidid                         youtubeid 
               "Neil_Abercrombie"                      "hawaiirep1" 
                             name                             title 
"Rep. Neil Abercrombie [D, HI-1]"                            "Rep." 
                            state                          district 
                             "HI"                               "1" 

\end{verbatim}

\item Get the value of a particular attribute (return NULL if no value specified)
\begin{verbatim}
 xmlGetAttr(root[[1]], 'state')
[1] "HI"
\end{verbatim}

\item Apply a function to each child of the root
\begin{verbatim}
xmlSApply(root,xmlGetAttr,'state')
\end{verbatim}



\end{itemize}


\section{Extract Data From XML Documents (Next Class)}
In XML document ``people.xml'', extract the following features and pack them in a data frame:
\begin{verbatim}
id
lastname
firstname
state
party
committee
\end{verbatim}
Note that the last item ``committee'' involves all codes of committees a congressman/woman belongs to.


\begin{verbatim}

rm(list=ls(all=TRUE))

## Function to extract selected attributes for each child of root
myproc <- function(x, attrs1, attrs2)
{
    ## Initialize the
    output <- NULL
    ## Number of attributes
    L1 <- length(attrs1)
    L2 <- length(attrs2)

    ## Extract selected attributes from <person ...>
    for (i in seq(1, L1))
    {
        if (!(is.null(xmlGetAttr(x, attrs1[i]))))
        {
            output <- c(output, xmlGetAttr(x, attrs1[i]))
        }else
        {
            output <- c(output, NA)
        }
    }

    ## Extract selected attributes from <role ...>
    ch <- xmlChildren(x)$role
    if (!is.null(ch))    ## Check if <role> exists
    {
        for (i in seq(1, L2))
        {
            if (!(is.null(xmlGetAttr(ch, attrs2[i]))))
            {
                output <- c(output, xmlGetAttr(ch, attrs2[i]))
            }else
            {
                output <- c(output, NA)
            }
        }
    }else
    {
        output <- c(output, NA, NA)
    }

    ## Extract selected attributes from <committee-assignment ...>
    num1 <- length(x)
    tmpout <- NULL
    flag <- 0    ## A flag indicating the existence of <committee-assignment ...>
    for (k in seq(1, num1))  ## Scan each child of <person>
    {
        if (xmlName(x[[k]]) == 'committee-assignment')
        {
            flag <- 1
            tmpout <- paste(tmpout, xmlGetAttr(x[[k]], 'code'))
        }
    }
    if (flag == 0)
    {
        tmpout <- 'NA'
    }
    output <- c(output, tmpout)
    return(output)
}


## Load the library and parse the XML file
library(XML)
doc <- xmlTreeParse('people.xml')
root <- xmlRoot(doc)

## Selected attributes in <person>
myattr1 <- c('id', 'lastname', 'firstname')
## Selected attributes in <role>
myattr2 <- c('state', 'party')

## Apply the function "myproc" and extract the attributes
tmp1 <- xmlSApply(root, myproc, myattr1, myattr2)
tmp1 <- t(tmp1)
rownames(tmp1) <- NULL
dt <- as.data.frame(tmp1)

#print(dt)

\end{verbatim}


\end{document}
