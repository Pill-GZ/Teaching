%\documentclass[letter,10.5pt]{article}
\documentclass[12pt]{article}
\usepackage{graphicx}
\usepackage{pdfpages}
\usepackage{array}
\usepackage{booktabs}
\usepackage{amsfonts,amssymb,amsmath,amsthm}
\usepackage{listings, boxedminipage}
\usepackage{bm}
%\usepackage{color}
\usepackage{url}
\usepackage{enumerate}

\newcommand{\rmnum}[1]{\romannumeral #1}
\newcommand{\Rmnum}[1]{\MakeUppercase{\romannumeral #1}}
\newcommand{\prob}{\mathbb{P}}
\newcommand{\E}{\mathbb{E}}
\newcommand{\Normal}{\mathsf{N}}
\newcommand{\iid}{\stackrel{\text{iid}}{\sim}}
\newcommand{\R}{\mathbb{R}}
\newcommand{\varx}{\mathbf{x}}
% grouping operators
\newcommand{\brac}[1]{\left[#1\right]}
\newcommand{\set}[1]{\left\{#1\right\}}
\newcommand{\abs}[1]{\left\lvert #1 \right\rvert}
\newcommand{\paren}[1]{\left(#1\right)}
\newcommand{\norm}[1]{\left\|#1\right\|}

\numberwithin{equation}{subsection}

\setlength{\textwidth}{150mm}
\setlength{\textheight}{230mm}
\setlength{\headheight}{-1.5cm}
\setlength{\topmargin}{-0.1cm}
\setlength{\oddsidemargin}{0cm}
\setlength{\evensidemargin}{0cm}
\setlength{\parskip}{1mm}
\setlength{\unitlength}{1mm}
\setlength{\parindent}{2.06em}
\pagestyle{plain}

\usepackage{color}
%\usepackage[dvipsnames]{xcolor}
\definecolor{codegreen}{rgb}{0,0.6,0}
\definecolor{codegray}{rgb}{0.5,0.5,0.5}
\definecolor{codepurple}{rgb}{0.58,0,0.82}
\definecolor{backcolour}{rgb}{0.95,0.95,0.92}
\lstdefinestyle{displaycode}{
	backgroundcolor=\color{backcolour},   
	commentstyle=\color{codegreen},
	keywordstyle=\color{magenta},
	numberstyle=\tiny\color{codegray},
	stringstyle=\color{codepurple},
	basicstyle=\footnotesize,
	breakatwhitespace=false,         
	breaklines=true,                 
	captionpos=b,                    
	keepspaces=true,                 
	columns=flexible,
	numbers=none,
	numbersep=5pt,                  
	showspaces=false,                
	showstringspaces=false,
	showtabs=false,                  
	tabsize=2%,
	%xleftmargin=2em,
	%xrightmargin=2em,
} 
\lstdefinestyle{displaycode2}{
	backgroundcolor=\color{backcolour},   
	commentstyle=\color{red},
	keywordstyle=\color{magenta},
	numberstyle=\tiny\color{codegray},
	stringstyle=\color{codepurple},
	basicstyle=\footnotesize,
	breakatwhitespace=false,         
	breaklines=true,                 
	captionpos=b,                    
	keepspaces=true,                 
	columns=flexible,
	numbers=none,
	numbersep=5pt,                  
	showspaces=false,                
	showstringspaces=false,
	showtabs=false,                  
	tabsize=2%,
	%xleftmargin=2em,
	%xrightmargin=2em,
} 

\lstset{style=displaycode}
%\newcommand{\displaycodefile}[1]{\lstinputlisting[language=R]{./codeblocks/#1.txt}}

\lstset{
	basicstyle=\ttfamily,
	mathescape
}


\title{\textbf{STAT 406 Lab 10 Handout}}
\author{Jun Guo}
\date{\today}
\begin{document}
\begin{center}
\large
\textbf{STAT 406 Lab 10 : Working with XML files}
\end{center}\vspace*{5mm}

\section{XML Basic Overview}
\begin{itemize}
	\item XML is widely used for exchanging information on the World Wide
Web. It stands for eXtensible Markup Language. 

\item Markup Language is a system of how the document is to be described or
logically structured, e.g., HTML and XML.
HTML is used for displaying data on the web browser,
XML is used for carrying data on the web.

\item The underpining data structure for XML files is of tree structure, called ``document tree''.

\item Structure of XML. 

Except for the file head declaration like 
\begin{lstlisting}[style=displaycode, language=XML]
<?xml version=�1.0�?>
\end{lstlisting} and comments in XML file like
\begin{lstlisting}[style=displaycode, language=XML]
<!-- This is a comment -->
\end{lstlisting}
The most concerned structure of an XML file in our class is in the main body of the file:

\item Elements(Nodes of the document tree) and Attributes: Elements contain the actual data of XML document. Elements
must have a start-tag and an end-tag which contain the element�s
name. The content sits between these two tags. 

\textbf{Elements have a
tree-based data structure.} \newline
The general tree structure of a XML file looks like below:

\begin{lstlisting}[style=displaycode, language=XML]
<root>
  <child>
    <subchild>.....</subchild>
  </child>
</root>
\end{lstlisting}

The following is a simple XML file, scan through it and answer the questions:
\begin{lstlisting}[style=displaycode, language=XML]
<?xml version="1.0" encoding="UTF-8"?>
<bookstore>
  <book category="children">
    <title lang="en">Harry Potter</title>
    <author>J K. Rowling</author>
    <year>2005</year>
    <price>29.99</price>
  </book>
  <book category="web">
    <title lang="en">Learning XML</title>
    <author>Erik T. Ray</author>
    <year>2003</year>
    <price>39.95</price>
  </book>
</bookstore>
\end{lstlisting}

\begin{enumerate}[1.] 
\item How many root nodes does this XML file have? What is the name of the root node? 
\item What are the the children nodes of the the root node? How many children nodes does the root have?
\item Please draw the document tree for this XML file.
\item What are the attribute and its value of the first ``book'' node? How about attribute and its value of the subchild node ``title''?
\item What is the value of the node ``author'' for the second book? 
\end{enumerate}

Notice that unlike the value of the nodes, \textbf{the value of attributes must be quoted}, using either single quotes or double quotes, like the following.
\noindent
\begin{lstlisting}[style=displaycode, language=XML]
<node_name attribute_name = "attribute value">  Node_value  </node_name>
\end{lstlisting}

\item Parse XML data: A parser is a component  of the complier which checks for correct syntax and builds a data structure. \textcolor{blue}{The R package XML provides the required parser and extends R�s capability in processing
XML files.}

\textbf{Download and save the XML file ``filename.xml'' in your R working directory first or nothing can be done even with correct codes}.
\begin{lstlisting}[style=displaycode, language=XML]
## Install and quote the R package XML first ##
if(!(require(RSQLite))){
	install.packages("XML", dep=T);
	require(XML);
}
doc <- xmlTreeParse('filename.xml') 
# doc contains the underpining tree structure of the file
\end{lstlisting}
\end{itemize}

\noindent
In the following two examples, we will parse and process the NSF award XML file and the US congress people XML file respectively.

\section{Examples of Processing XML files}

\subsection{Example I. NSF Award files}

\noindent First we want to parse a .xml file to reveal the document tree structure. Download the \textbf{award1.xml} file in ctools and save to your working directory in R.
\begin{itemize}
\item Parse the file:
\begin{verbatim}
doctree = xmlParse(`award1.xml');
\end{verbatim}


\item Extracting the root:
\begin{verbatim}
root <- xmlRoot(doctree);
xmlName(root);
# [1] "rootTag"
\end{verbatim}

\item List the children nodes for the root, how many are there and what are the names?
\begin{verbatim}
rtCh = xmlChildren(root);
length(rtCh);
# [1] 1
names(rtCh);
# [1] "Award"
\end{verbatim}

\item Extract all the children of the first child of the root:
\begin{verbatim}
ch <- xmlChildren(root[[1]]); 
head(ch, 3);
# $AwardTitle ...
  $AwardEffectiveDate ... 
  $AwardExpirationDate ...
length(ch);
# [1] 16
names(ch);
# [1] "AwardTitle"         
  [2] "AwardEffectiveDate" 
  [3] "AwardExpirationDate"
  [4] "AwardAmount"
  ...
\end{verbatim}
\end{itemize}
\noindent 
Once we understand the document structure, we will pull out the information we need.\newline

\textbf{Question 1} : Download and save the 2016 folder in ctools to your working directory, extract with R the \textbf{title, start/end dates, amount, investigator's name and the institution name} for each award in the 2016 award folder and output these information in a data frame. Each XML file in the NSF award folder 2016 records one award. \newline

\textbf{Solution 1} : \textcolor{green}{See lab10.r code.}

\subsection{Example II. US congress people.xml file}

\begin{itemize}
\item Parse and explore the document tree structure as usual.
\begin{verbatim}
doc = xmlTreeParse("people.xml");
root = xmlRoot(doc); 
ls_people = xmlChildren(root); length(ls_people);

person1 = root[[1]]; 
print(person1);
# use xmlAttr() to list the attributes, 
# and xmlGetAttr() to get each attribute's value
xmlAttr(person1);
xmlGetAttr(person1, 'id');
# list all children nodes for person1;
ls_person1 = xmlChildren(person1);
names(ls_person); # only 1 'role' child node

person2 = root[[2]]; 
print(person2);
ls_person2 = xmlChildren(person2);
names(ls_person2); # 3 children nodes
# get the first committee assignment subchild for person2
ls_person2$"committee-assignment";
\end{verbatim}
\end{itemize}

\noindent
After the exploration, we begin processing the file. \newline

\noindent
\textbf{Question 2} : Get the following attributes of nodes in people.xml file, they are 'id','lastname','firstname','gender','party','state','committee'.\newline

\noindent
\textbf{Solution 2} : \textcolor{green}{See lab10.r code.}


























\end{document}