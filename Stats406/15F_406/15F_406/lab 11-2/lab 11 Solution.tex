\documentclass[12pt]{article}
\setlength{\oddsidemargin}{0.0in} \setlength{\evensidemargin}{0.0in}
\setlength{\textwidth}{6.5in} \setlength{\topmargin}{-0.25in}
\setlength{\textheight}{8.75in}
\usepackage{amsmath, amssymb, amsfonts, amscd, xspace, pifont, natbib}
\usepackage{epsfig, amsfonts, verbatim, multirow}
\usepackage{epstopdf}
%\usepackage{setspace}
\newcommand{\mycite}[1]{{\citeNP{#1}}}

%%%%%%%%%%%%%%%%%%%%%%%%%%%%%%%%%%%%%%%%%%%%%%%%%%%%%%%%%%%%%%%%%%%%%%%%%%%%%%%%%%%%%%%%%%%%%%%%%%%
% Different font in captions
\newcommand{\captionfonts}{\small}
\makeatletter  % Allow the use of @ in command names
\long\def\@makecaption#1#2{%
  \vskip\abovecaptionskip
  \sbox\@tempboxa{{\captionfonts #1: #2}}%
  \ifdim \wd\@tempboxa >\hsize
    {\captionfonts #1: #2\par}
  \else
    \hbox to\hsize{\hfil\box\@tempboxa\hfil}%
  \fi
  \vskip\belowcaptionskip}
\makeatother   % Cancel the effect of \makeatletter
%%%%%%%%%%%%%%%%%%%%%%%%%%%%%%%%%%%%%%%%%%%%%%%%%%%%%%%%%%%%%%%%%%%%%%%%%%%%%%%%%%%%%%%%%%%%%%%%%%%

%% Matrix, Vector
\newcommand{\V}[1]{\ensuremath{\boldsymbol{#1}}\xspace}
\newcommand{\M}[1]{\ensuremath{\boldsymbol{#1}}\xspace}
%% Math Functions
\newcommand{\F}[1]{\ensuremath{\mathrm{#1}}\xspace}
\newcommand{\sgn}{\F{sgn}}
\newcommand{\tr}{\F{trace}}
\newcommand{\diag}{\F{diag}}
\newcommand{\dett}{\F{det}}
%% Transpose
\newcommand{\T}[1]{\ensuremath{{#1}^{\mbox{\sf\tiny T}}}}

%%
\def\bX{\boldsymbol X}
\def\bY{\boldsymbol Y}
\def\bbeta{\boldsymbol \beta}
\def\blambda{\boldsymbol \lambda}
\def\bepsilon{\boldsymbol \epsilon}
\def\bone{\boldsymbol{1}}
\def\bzero{\boldsymbol 0}
\def\E{\mbox{E}}
\def\var{\mbox{var}}
\def\gauss{\mbox{N}}
\def\lap{\mbox{L}}
\def\G{\mbox{G}}
\def\go{\rightarrow}
\def\invG{\mbox{G}^{-1}}
\def\argmin{\arg\min}


\newtheorem{theorem}{Theorem}
\newtheorem{proposition}{Proposition}
\newtheorem{lemma}{Lemma}
\newtheorem{corollary}{Corollary}
\newtheorem{algorithm}{Algorithm}
\newtheorem{definition}{Definition}
\newtheorem{remark}{Remark}


\begin{document}

%\baselineskip = 1.\baselineskip
%\baselineskip = 2.0\baselineskip
%\doublespacing

%\begin{titlepage}
\title{\Large \bf STATS 406F15 Lab 11}
\date{}

\maketitle
%\bigskip
%\bigskip

%\newpage
%\par\noindent
%{\sc Summary. \\

%\bigskip
%\medskip
%%%%%%%%%%%%%%%%%%%%%%%%%%%%%%%%%%%%%%%%%%%%%%%%%%%%%%%%%%%555555555555555555555555555555555555555555555555555555555555555555555555555555555555555
%\par\noindent
%GLasso; Precision matrix; Spectral clustering; Commute similarity.

%\thispagestyle{empty}
%\end{titlepage}

%\setcounter{equation}{0}

\section{Optimization using the optim function}

\subsection*{Problem 1}

Suppose $X_1, \ldots, X_n$ are i.i.d. $N(\mu, \sigma^2)$, where both $\mu$ and $\sigma^2$ are unknown. We will compute the MLE for parameters $(\mu, \sigma^2)$. The log-likelihood is:
$$l(\mu, \sigma^2) = -\frac{n}{2} \log(2 \pi ) -\frac{1}{2} (n \log(\sigma^2 ) + \frac{1}{\sigma^2} \sum_{i=1}^n (x_i - \mu)^2) \ .$$
Generate data with $\mu=2$ and $\sigma^2=4$.

\subsection*{Solution}

\begin{verbatim}

## Simulate 100 data points from Normal distribution with mu=2, sigma^2=4
X <- rnorm(100, mean=2, sd=sqrt(4))
n <- 100

# likelihood t[1]: mu, t[2]: sigma^2
# if sigma < 0 return Inf, instead of NaN.
f <- function(t)
{print(X)
    if( t[2] > 0)
    {
        return( -sum( dnorm(X, mean=t[1], sd=sqrt(t[2]), log=TRUE) ) )
    } else
    {
        return(Inf)
    }
}

# gradient function
df <- function(t)
{
    mu <- t[1]; sig <- t[2];
    g <- rep(0,2)
    g[1] <- (1/sig) * sum(X - mu)
    g[2] <- (-n/(2*sig)) + sum( (X-mu)^2 )/(2*sig^2)
    return(-g)
}

## Run optim() 
objoptim <- optim(par=c(0, 1), fn=f, gr=df, method='BFGS')
print(objoptim$par)

\end{verbatim}

\section{Newton-Raphson methods}
\subsection*{Newton's method for solving $f(x)=0$}

$$
y=f(x_{0})+f'(x_{0})(x-x_{0})
$$
The tangent line crosses $0$ when $y=0$, which happens when
$$
f(x_{0})=-f'(x_{0})(x-x_{0})
$$
rearranging terms approporiately, we find that
$$
x=x_{0}-f(x_{0})/f'(x_{0})
$$
Now setting $x_{0}=x$ and repeating until convergence is essentially Newton $\mathrm{s}$ method. We can see graphically how this works with the function: $f(x)=e^{x}-5$

\subsection*{Solution}

\begin{verbatim}
# f and f'
f <- function(x) exp(x)-5
df <- function(x) exp(x)
# make the initial plot
v <- seq(0,4,length=1000)
plot(v,f(v),type="l",col=8)
abline(h=0)
# start point
x0 <- 3.5
# paste this repeatedly to see Newton's method work
segments(x0,0,x0,f(x0),col=2,lty=3)
slope <- df(x0)
g <- function(x) f(x0) + slope*(x - x0)
v = seq(x0 - f(x0)/df(x0),x0,length=1000)
lines(v,g(v),lty=3,col=4)
x0 <- x0 - f(x0)/df(x0)
\end{verbatim}

\subsection*{Newton's method for finding where $f(x)$ is at its maximum}

We can write this as a generic function in R as follows.
\begin{verbatim}
# f: the function you want the max of
# df: derivative of f; d2f: second derivative
# x0: start value, tol: convergence criterion
# maxit: max # of iterations before it is considered to have failed
newton <- function(x0, f, df, d2f, tol=1e-4, pr=FALSE){
  # iteration counter
  k <- 0
  # initial function, derivatives, and x values
  fval <- f(x0)
  grad <- df(x0)
  hess <- d2f(x0)
  xk_1 <- x0
  cond1 <- sqrt(sum(grad^2))
  cond2 <- Inf
  # see if the starting value is already close enough
  if( (cond1 < tol) ) return(x0)
  while( (cond1 > tol) & (cond2 > tol) )
  {
    L <- 1
    bool <- TRUE
    while(bool == TRUE){
      xk <- xk_1 - L * solve(hess) %*% grad
      # see if we've found an uphill step
      if( f(xk) > fval ){
        bool = FALSE
        grad <- df(xk)
        fval <- f(xk)
        hess <- d2f(xk)
        # make the stepsize a little smaller
      } 
      else {
        L = L/2
        if( abs(L) < 1e-20 ) return("Failed to find uphill step - try new start val")
      }
    }
    # calculate convergence criteria
    cond1 <- sqrt( sum(grad^2) )
    cond2 <- sqrt( sum( (xk-xk_1)^2 ))/(tol + sqrt(sum(xk^2)))
    # add to counter and update x
    k <- k + 1
    xk_1 <- xk
  }
  if(pr == TRUE) print( sprintf("Took %i iterations", k) )
  return(xk)
}
\end{verbatim}

\subsection*{Problem 2}

Let $f(x)=e^{-(x^{2})}$. It is easy to see that $f'(x)=-2xf(x)$ and $f'(x)=-2f(x)-2xf'(x)$. 
Now we code each of these functions and call the newton $()$ function:
\subsection*{Solution}
\begin{verbatim}
f <- function(x) exp(-(x^2))
df <- function(x) -2*x*f(x)
d2f <- function(x) -2*f(x)-2*x*df(x)
# Newton's method starting at x0 = 2/3
newton(2/3, f, df, d2f, 1e-7)
\end{verbatim}


The maximum at $0$ appears to have been found.
\subsection*{Problem 3}
Suppose $X_{1}$, . . . , $X_{n}$ are iid $N(\mu,\ \sigma^{2})$ where both $\mu$ and $\sigma^{2}$ are unknown. We will write a program to do maximum likelihood estimation for $\theta=(\mu,\ \sigma^{2})$. The log-likelihood is
$$
l(\mu,\ \sigma^{2})=-\frac{1}{2}[n\ \log(2\pi\sigma^{2})+\frac{1}{\sigma^{2}}\sum_{i=1}^{n}(x_{i}-\mu)^{2}]
$$
so the elements of the gradient are
$$
\frac{\partial l}{\partial\mu}=\frac{1}{\sigma^{2}}\sum_{i=1}^{n}(x_{i}-\mu)
$$
and
$$
\frac{\partial l}{\partial\sigma^{2}}=-\frac{n}{2\sigma^{2}}+\frac{1}{2\sigma^{4}}\sum_{i=1}^{n}(x_{i}-\mu)^{2}
$$
finally, the elements of the hessian are
$$
\frac{\partial^{2}l}{\partial\mu^{2}}=-\frac{n}{\sigma^{2}},
$$
$$
\frac{\partial^{2}l}{\partial(\sigma^{2})^{2}}=\frac{n}{2\sigma^{4}}-\frac{1}{\sigma^{6}}\sum_{i=1}^{n}(x_{i}-\mu)^{2}
$$
and
$$
\frac{\partial^{2}l}{\partial\mu\partial\sigma^{2}}=-\frac{1}{\sigma^{4}}\sum_{i=1}^{n}(x_{i}-\mu)
$$
We can code these functions easily in $\mathrm{R}$ and call newton to get the MLEs
\begin{verbatim}
X <- rnorm(100, mean=2, sd=sqrt(4))
n <- 100
# likelihood t[1]: mu, t[2]: sigma^2
# if sigma < 0 return -Inf, instead of NaN.
f <- function(t){
  if( t[2] > 0)
  { return( sum( dnorm(X, mean=t[1], sd=sqrt(t[2]), log=TRUE) ) )
  } 
  else {
	  return(-Inf)
  }
}
# score function
df <- function(t){
  mu <- t[1]; sig <- t[2];
  g <- rep(0,2)
  g[1] <- (1/sig) * sum(X - mu)
  g[2] <- (-n/(2*sig)) + sum( (X-mu)^2 )/(2*sig^2)
  return(g)
}
# hessian
d2f <- function(t){
  mu <- t[1]; sig <- t[2];
  h <- matrix(0,2,2)
  h[1,1] <- -n/sig
  h[2,2] <- (n/(2*sig^2)) - sum( (X-mu)^2 )/(sig^3)
  h[1,2] <- -sum( (X-mu) )/(sig^2)
  h[2,1] <- h[1,2]
  return(h)
}
newton( c(0,1), f, df, d2f)

\end{verbatim}


\section{EM algorithm}
\subsection*{Example}
Consider a simple regression model $Y_{i}=\beta x_{i}+\epsilon_{i}$, where $\epsilon_{i}\sim N(0,\ 1)$, for $ i=1\ldots n+m$. 
Suppose that the first $n$ data $Y_{i}$ are observed while the last $m$ are censored at some positive threshold $C$. 
Censoring means that all we know is that $Y_{i}>C$, but we do not actually observe $Y_{i}$. 
We wish to estimate $\beta$. If we have seen all the $Y_{i}$, then the complete likelihood is
$$
l_{com}(\beta|Y)=-\frac{\beta^{2}}{2}\sum_{i=1}^{n+m}x_{i}^{2}+\beta\sum_{i=1}^{n+m}x_{i}y_{i}+const.
$$
Therefore the estimate of $\beta$ in complete data setting is $[\displaystyle \sum_{i=1}^{n+m}x_{i}y_{i}]/[\sum_{i=1}^{n+m}x_{i}^{2}]$. With censoring, all we see on the last $m$ samples is that $I_{i}=1$, where $I_{i}=1$ if $Y_{i}>C$, and $I_{i}=0$ otherwise. In this problem the complete likelihood depends only linearly on $Y_{i}$. So to implement the EM, we only need to find $E(Y_{i}|I_{i}=1)$, for a given value of $\beta$. Since $Z_{i}=Y_{i}-x_{i}\beta\sim N(0,1)$, we have $E(Y_{i}|I_{i}=1)=x_{i}\beta+E(Z_{i}|I_{i}=1)$. Further

$E(Z_{i}|I_{i}=1)=E(Y_{i}-x_{i}\displaystyle \beta|Y_{i}-x_{i}\beta>C-x_{i}\beta)=E(Z_{i}|Z_{i}>C-x_{i}\beta)=\frac{\int_{C-x_{i}\beta}^{\infty}z\phi(z)dz}{1-\Phi(C-x_{i}\beta)}$

$$
E(Y_{i}|I_{i}=1)=x_{i}\beta+\frac{\phi(C-x_{i}\beta)}{1-\Phi(C-x_{i}\beta)}=m_{i}(\beta).
$$
Given $\beta_{k}$, if we replace the $Y_{i}$ for $n+1\leq i\leq n+m$ by $m_{i}(\beta_{k})$, we obtain the $Q$ function
$$
Q(\beta|\beta_{k})=-\frac{\beta^{2}}{2}\sum_{i=1}^{n+m}x_{i}^{2}+\beta\sum_{i=1}^{n}x_{i}y_{i}+\beta\sum_{i=n+1}^{n+m}x_{i}m_{i}+const.
$$
Maximizing this function gives easily the solution
$$
\frac{\sum_{i=1}^{n}x_{i}y_{i}+\sum_{i=n+1}^{n+m}x_{i}m_{i}}{\sum_{i=1}^{n+m_{\chi_{i}^{2}}}}
$$
We obtain the following EM algorithm for estimating the mle of $\beta$.

Algorithm (The EM algorithm for the censored data model).

( $\mathrm{E}$ step): At stage $k$, given $\beta_{k}$ a current guess of $\beta$, compute $m_{i}=x_{i}\beta_{k}+\phi(C-x_{i}\beta_{k})/[1-\Phi(C-x_{i}\beta_{k})]$, for $i=(n+1),\ \ldots,\ (n+m)$.

( $\mathrm{M}$ step): Compute the new estimate of $\beta$,
$$
\beta_{k+1}=\frac{\sum_{i=1}^{n}x_{i}y_{i}+\sum_{i=n+1}^{n+m}x_{i}m_{i}}{\sum_{i=1}^{n+m}x_{i}^2}
$$
\subsection*{Solution}

\begin{verbatim}
############
beta_star=2; n_0=100
X=rnorm(n_0,0,1)
Y=rnorm(n_0,mean=beta_star*X,sd=1)
C=2.5
n=sum(Y<=C); m=sum(Y>C)
X=c(X[Y<=C],X[Y>C])
Y=Y[Y<=C]
beta=0
K=50
Res=double(K)
for (k in 1:K){
  m_beta= X[(n+1):(n+m)]*beta 
  + dnorm(C-X[(n+1):(n+m)]*beta)/(1-pnorm(C-X[(n+1):(n+m)]*beta))
  beta =( sum(X[1:n]*Y) + sum(X[(n+1):(n+m)]*m_beta) ) / ( sum(X^2) )
  Res[k]=beta
}
plot(Res,type='b',col='blue')

\end{verbatim}


\end{document}
