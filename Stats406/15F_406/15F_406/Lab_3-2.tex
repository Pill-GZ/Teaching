\documentclass[10pt]{article}
\setlength{\oddsidemargin}{0.0in} \setlength{\evensidemargin}{0.0in}
\setlength{\textwidth}{6.5in} \setlength{\topmargin}{-0.25in}
\setlength{\textheight}{8.75in}
\usepackage{amsmath, amssymb, amsfonts, amscd, xspace, pifont, natbib, fullpage}
\usepackage{epsfig, amsfonts, verbatim, multirow, hyperref}
\usepackage{epstopdf}
\usepackage{listings}
\lstset{
  basicstyle=\ttfamily,
  mathescape
}
%\usepackage{setspace}
\newcommand{\mycite}[1]{{\citeNP{#1}}}

\parindent=0pt

%%%%%%%%%%%%%%%%%%%%%%%%%%%%%%%%%%%%%%%%%%%%%%%%%%%%%%%%%%%%%%%%%%%%%%%%%%%%%%%%%%%%%%%%%%%%%%%%%%%
% Different font in captions
\newcommand{\captionfonts}{\small}
\makeatletter  % Allow the use of @ in command names
\long\def\@makecaption#1#2{%
  \vskip\abovecaptionskip
  \sbox\@tempboxa{{\captionfonts #1: #2}}%
  \ifdim \wd\@tempboxa >\hsize
    {\captionfonts #1: #2\par}
  \else
    \hbox to\hsize{\hfil\box\@tempboxa\hfil}%
  \fi
  \vskip\belowcaptionskip}
\makeatother   % Cancel the effect of \makeatletter
%%%%%%%%%%%%%%%%%%%%%%%%%%%%%%%%%%%%%%%%%%%%%%%%%%%%%%%%%%%%%%%%%%%%%%%%%%%%%%%%%%%%%%%%%%%%%%%%%%%

%% Matrix, Vector
\newcommand{\V}[1]{\ensuremath{\boldsymbol{#1}}\xspace}
\newcommand{\M}[1]{\ensuremath{\boldsymbol{#1}}\xspace}
%% Math Functions
\newcommand{\F}[1]{\ensuremath{\mathrm{#1}}\xspace}
\newcommand{\sgn}{\F{sgn}}
\newcommand{\tr}{\F{trace}}
\newcommand{\diag}{\F{diag}}
\newcommand{\dett}{\F{det}}
%% Transpose
\newcommand{\T}[1]{\ensuremath{{#1}^{\mbox{\sf\tiny T}}}}

%%
\def  \R  {\boldsymbol R}
\def\bX{\boldsymbol X}
\def\bY{\boldsymbol Y}
\def\bbeta{\boldsymbol \beta}
\def\blambda{\boldsymbol \lambda}
\def\bepsilon{\boldsymbol \epsilon}
\def\bone{\boldsymbol{1}}
\def\bzero{\boldsymbol 0}
\def\E{\mbox{E}}
\def\var{\mbox{var}}
\def\gauss{\mbox{N}}
\def\lap{\mbox{L}}
\def\G{\mbox{G}}
\def\go{  $\R$  ightarrow}
\def\invG{\mbox{G}^{-1}}
\def\argmin{\arg\min}

\newtheorem{theorem}{Theorem}
\newtheorem{proposition}{Proposition}
\newtheorem{lemma}{Lemma}
\newtheorem{corollary}{Corollary}
\newtheorem{algorithm}{Algorithm}
\newtheorem{definition}{Definition}
\newtheorem{remark}{Remark}

\usepackage{color}
\definecolor{codegreen}{rgb}{0,0.6,0}
\definecolor{codegray}{rgb}{0.5,0.5,0.5}
\definecolor{codepurple}{rgb}{0.58,0,0.82}
\definecolor{backcolour}{rgb}{0.95,0.95,0.92}
\lstdefinestyle{displaycode}{
    backgroundcolor=\color{backcolour},   
    commentstyle=\color{codegreen},
    keywordstyle=\color{magenta},
    numberstyle=\tiny\color{codegray},
    stringstyle=\color{codepurple},
    basicstyle=\footnotesize,
    breakatwhitespace=false,         
    breaklines=true,                 
    captionpos=b,                    
    keepspaces=true,                 
    columns=flexible,
    numbers=none,
    numbersep=5pt,                  
    showspaces=false,                
    showstringspaces=true,
    showtabs=false,                  
    tabsize=2%,
    %xleftmargin=2em,
    %xrightmargin=2em,
} 

\lstset{style=displaycode}
%\newcommand{\displaycodefile}[1]{\lstinputlisting[language=R]{./codeblocks/#1.txt}}

\begin{document}

\title{\Large \bf STATS 406 F15: Lab 03}
\date{}

\maketitle


1. Plotting: making your plots prefessional
1.1 Brief recap of basic examples: plot, boxplot, hist
1.2 Screen management:
* par(mfrow=c(n1, n2)) or mfcol
* layout(matrix): can control the order of plot, can split into different sizes
* (will not elaborate)screen.split
1.3 Contour plots
* contour(x, y, z)
* lattice::levelplot
* image(x, y, z)
1.4 Frequently used plotting parameters
* line/point styles: type, lty, col, pch: scalar or vector
* plot range: xlim, ylim
* texts: main, sub, xlab, ylab -- can use paste/sprintf to include variables
* font size: cex(default), cex.main/sub/lab/axis
texts can be stand alone using title command or placed within plot
1.5 Low-level plots: adding something to existing plot
* point
* abline
* floating text
* legend
**  Including legend is a basic courtsey in many formal write-ups
**  legend(location, legendnames, style)
**  Principle: if set style parameters for the plot command, use consistent configuration for legend
* Remark: the usage of expression in texts

2. Simplest asymptotics: LLNo's and CLT
1.1 Illustration of SLLNo's and WLLNo's
1.2 Conceptual clarification of CLT: what is converging to a standard normal distribution?
* To illustrate CLT, we need many samples


\section{Plotting: making your plots prefessional}

\subsection{Recap: basic plot commands}
\begin{itemize}
	\item {\bf plot()}. Here we only recap list plots. In practice you rarely work with function plots in R.
\begin{lstlisting}[style=displaycode, language=R]
	# Recap: plot()
	a_vec = seq(from=0, to=3, by=0.1);
	b_vec = a_vec + rnorm(length(a_vec));
	plot(b_vec~a_vec);
\end{lstlisting}
	Looks awkward. Today's lab is dedicated to improve it.
	\item {\bf hist()}.
\begin{lstlisting}[style=displaycode, language=R]
	# Recap: hist()
	# continued from the previous example...
	hist(b_vec);
\end{lstlisting}
	\item {\bf boxplot()}.
\begin{lstlisting}[style=displaycode, language=R]
	# Recap: boxplot();
	# 3 groups of data, each group contains 20 random numbers
	groupsize = 20; ngroup = 3;
	a_vec = rnorm(ngroup*groupsize); a_vec = a_vec + seq(from=0, to=5, length.out=length(a_vec));
	b_vec = as.factor( rep(letters[1:ngroup], rep(groupsize, ngroup)) );
	# side-by-side group-wise boxplots
	test_data_frame = data.frame(a_vec, b_vec);
	boxplot( with(test_data_frame, a_vec ~ b_vec) );
\end{lstlisting}
\end{itemize}



\subsection{Manage screen layout}
\begin{itemize}
	\item Method 1: {\bf par(mfrow=c(n1, n2))}(by row) or {\bf par(mfcol=c(n1, n2))}(by column).
\begin{lstlisting}[style=displaycode, language=R]
	# par(mfrow=...)
	# Open a new screen.
	x11(); # Not applicable under commandline
	# Split screen
	par(mfrow = c(2, 1)); 
	# Plots
	set.seed(2015);
	plot(rnorm(1000));
	qqnorm(runif(1000));
	#
	# When we're done:
	dev.off(); # or click to close the window
\end{lstlisting}
	Drawback: always evenly split the screen, cannot control screen block sizes.
	\item Method 2: {\bf layout(splitmatrix)}. More flexible.\\
	\underline{Example:} If we want a big square plot at the top-left corner and split the rest accordingly, first set the split matrix that looks like:
	\begin{equation*}
		\textrm{split matrix} = 
		\begin{pmatrix}
		1&1&1&2&2\\
		1&1&1&2&2\\
		1&1&1&2&2\\
		3&3&3&4&4\\
		3&3&3&4&4
		\end{pmatrix}
	\end{equation*}
\begin{lstlisting}[style=displaycode, language=R]
	# layout(splitmatrix)
	x11();
	# Prepare the split matrix
	SplitMat = array(0, c(5, 5));
	SplitMat[1:3, 1:3] = 1;
	SplitMat[4:5, 1:3] = 2; SplitMat[1:3, 4:5] = 3;
	SplitMat[4:5, 4:5] = 4;
\end{lstlisting}
	and then split the screen accordingly and plot:
\begin{lstlisting}[style=displaycode, language=R]
	# Split the screen
	layout(SplitMat, 2, 2);
	layout.show(4); # just to confirm, not required
	# Plots
	set.seed(2015);
	plot(rnorm(1000));
	qqnorm(runif(1000));
	qqnorm(rt(1000, df=3));
	hist(rnorm(1000));
	#
	# When we're done:
	dev.off(); # or click to close the window
\end{lstlisting}
	\item Method 3: {\bf split.screen()}. Will not elaborate. Check documentation if interested.
\end{itemize}


\subsection{More plotting commands: contour plots}
Apart from line plots, another important family of plots in R are contour plots.
\begin{itemize}
	\item {\bf contour()}. Shows only contour lines.
\begin{lstlisting}[style=displaycode, language=R]
	# A naive contour plot
	# Prepare an integer multiply chart within 100
	x_mat = (1:10) %*% t(rep(1, 10));
	y_mat = t(x_mat);
	z_mat = x_mat*y_mat;
	# Contour plot
	contour(z_mat);
\end{lstlisting}
	\item {\bf levelplot(). Requires package ``lattice''}. If we want to use colors to indicate entry values...
\begin{lstlisting}[style=displaycode, language=R]
	# Continuing the last example
	require(lattice);
	print(levelplot(z_mat)); # Always use print with levelplot.
\end{lstlisting}
	\item {\bf image()}. Similar to {\bf levelplot()}.
\begin{lstlisting}[style=displaycode, language=R]
# Continuing the last example
	image(z_mat);
\end{lstlisting}	
\end{itemize}



\subsection{Adjusting plot parameters}
Plots almost never appear ``raw''(or ``naked'') in formal write-ups. Now we learn how to tune the appearance of our plots. To embed our discussion in a specific context, let's recall the light bulb example we went through last week:

\noindent\fbox{%
	\parbox{\textwidth}{%
		Example: a simple on-off Markov chain\\
		Each light bulb has two status: 1=lighted up or 0=burnt out.  In each round, a light bulb goes from status 1 to 0 with probability $q=0.05$ or from status 0 to 1 with probability $p=0.10$.
	}%
}
This time we focus the update track of a light bulb. First, prepare data for plotting:
\begin{lstlisting}[style=displaycode, language=R]
# Illustrate the update track of one light bulb
# Prepare data
set.seed(2015);
p = 0.10; q = 0.05;
nround = 2000;
UpdateTrack = integer(nround);
UpdateTrack[1] = 1;
for(i in 2:nround){
	if(UpdateTrack[i-1]==0){
		UpdateTrack[i] = rbinom(1, 1, p);
	} else{
		UpdateTrack[i] = rbinom(1, 1, 1-q);
	}
}
\end{lstlisting}

Then we plot it:
\begin{lstlisting}[style=displaycode, language=R]
	# Plot
	plot(UpdateTrack,
		type='b', lty=3, pch=25, col='blue',
		xlim=c(-1, 52), ylim=c(-0.2, 1.2),
		main=paste('Light bulb status update within ', nround, ' rounds', sep=''), sub='An on-off process illustration', xlab='Index', ylab='Status',
		cex.lab=1.5, cex.main=1.25, cex.sub=1.25, # col.lab='darkblue',
		axes=FALSE, # to be specified later
	)
	axis(1, at=c(1, 25, 50), cex.axis=1.5);
	axis(2, at=c(0, 1), cex.axis=1.5);
	box();
\end{lstlisting}
This example contains a few frequently used plotting parameters:
	\begin{itemize}
		\item Line/Point styles: {\bf type}, {\bf lty}, {\bf pch}, {\bf col}, ...
		\begin{itemize}
			\item {\bf type}: consult {\bf ?plot}, it refers to the type of the line. `b' for showing both the points and the line, `o' for both but over-positioning each other, `l' for line-only and `p' for points-only.
			\item {\bf lty}: line type, take integer values.
			\item {\bf pch}: the point marker used, take integer or character values.
			\item {\bf col}: color, take integer or character values.
		\end{itemize}
		\item Plot range: {\bf xlim}, {\bf ylim}. Here we expanded the plot ranges to have more comfortable margins.
		\item Texts: {\bf main}, {\bf sub}, {\bf xlab}, {\bf ylab}, ...
		\begin{itemize}
			\item {\bf main}: main title of the plot.
			\item {\bf sub}: subtitle.
			\item {\bf xlim}, {\bf ylim}: ranges of axes.
		\end{itemize}
		\item Font sizes: {\bf cex}(default), {\bf cex.main}, {\bf cex.sub}, {\bf cex.lab}, {\bf cex.axis}, ...
	\end{itemize}
	{\bf Quiz:} How to assign different colors to different points?



\subsection{Low-level plots}
We can add additional plotting objects like points or lines, as well as texts to an existing plot. Such additions are called low-level plots.
\begin{itemize}
	\item Adding points: {\bf points(x, y)}.
\begin{lstlisting}[style=displaycode, language=R]
	# Continuing the last example
	Points_x = c(13,15,17,30,31,39);
	Points_y = c(0.3, 0.4, 0.5, 0.6, 0.7, 0.8);
	points(x, y);
\end{lstlisting}
	\item Adding lines: {\bf lines(x, y)}, similar to {\bf points}, omitted here. Check documentation.
	\item Adding a reference line: {\bf abline(a=intercept, b=slope)}.
\begin{lstlisting}[style=displaycode, language=R]
	# Continuing the last example
	abline(0.2, 0.5/nround, type=3, col='lightblue');
\end{lstlisting}
	\item Adding a legend: {\bf legend()}
	\begin{itemize}
		\item Usage: {\bf legend(location, legendnames, optimalstyle)}.
		\item The style options should be consistent with curves.
	\end{itemize}
\begin{lstlisting}[style=displaycode, language=R]
	# Continuing the last example
	legend('topright', c('LightBulb 1'), lty=3, pch=25, col=c('blue'));
\end{lstlisting}
\end{itemize}


\subsection{Exercises}
See Lab\_3.R.


\section{Law of large numbers and central limit theorem}

\subsection{Illustrations of strong- and weak- laws of large numbers}
\begin{lstlisting}[style=displaycode, language=R]
	### Illustrations of SLLNo's and WLLNo's
	nsamples = 1000; samplesize = 1000;
	BigRVMatrix = matrix(runif(nsamples*samplesize), c(nsamples, samplesize));
\end{lstlisting}

\subsection{Central limit theorem}
In a random sample $S = \{ X_1, X_2,\ldots, X_n \}$, ...
\begin{itemize}
	\item {\bf Whose distribution is asymptotically normal?}
	\item {\bf Does the sample distribution go to normal?}
	\item {\bf Compare CLT under different parent population distributions}
\begin{lstlisting}[style=displaycode, language=R]
	# See Lab_3.R
\end{lstlisting}
\end{itemize}



\end{document}
