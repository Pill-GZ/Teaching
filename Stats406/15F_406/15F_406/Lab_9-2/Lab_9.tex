\documentclass[12pt]{article}
\setlength{\oddsidemargin}{0.0in} \setlength{\evensidemargin}{0.0in}
\setlength{\textwidth}{6.5in} \setlength{\topmargin}{-0.25in}
\setlength{\textheight}{8.75in}
\usepackage{amsmath, amssymb, amsfonts, amsthm, amscd, xspace, pifont, natbib, fullpage, enumitem, bm, bbm}
\usepackage{fullpage}
\usepackage{graphicx, float}
\usepackage{epsfig, amsfonts, verbatim, multirow, hyperref}
\usepackage{epstopdf}
\usepackage{listings, boxedminipage}

\lstset{
	basicstyle=\ttfamily,
	mathescape
}
%\usepackage{setspace}
\newcommand{\mycite}[1]{{\citeNP{#1}}}

\parindent=0pt

%%%%%%%%%%%%%%%%%%%%%%%%%%%%%%%%%%%%%%%%%%%%%%%%%%%%%%%%%%%%%%%%%%%%%%%%%%%%%%%%%%%%%%%%%%%%%%%%%%%
% Different font in captions
\newcommand{\captionfonts}{\small}
\makeatletter  % Allow the use of @ in command names
\long\def\@makecaption#1#2{%
	\vskip\abovecaptionskip
	\sbox\@tempboxa{{\captionfonts #1: #2}}%
	\ifdim \wd\@tempboxa >\hsize
	{\captionfonts #1: #2\par}
	\else
	\hbox to\hsize{\hfil\box\@tempboxa\hfil}%
	\fi
	\vskip\belowcaptionskip}
\makeatother   % Cancel the effect of \makeatletter
%%%%%%%%%%%%%%%%%%%%%%%%%%%%%%%%%%%%%%%%%%%%%%%%%%%%%%%%%%%%%%%%%%%%%%%%%%%%%%%%%%%%%%%%%%%%%%%%%%%

%% Matrix, Vector
\newcommand{\V}[1]{\ensuremath{\boldsymbol{#1}}\xspace}
\newcommand{\M}[1]{\ensuremath{\boldsymbol{#1}}\xspace}
%% Math Functions
\newcommand{\F}[1]{\ensuremath{\mathrm{#1}}\xspace}
\newcommand{\sgn}{\F{sgn}}
\newcommand{\tr}{\F{trace}}
\newcommand{\diag}{\F{diag}}
\newcommand{\dett}{\F{det}}
%% Transpose
\newcommand{\T}[1]{\ensuremath{{#1}^{\mbox{\sf\tiny T}}}}

%%
\def  \R  {\boldsymbol R}
\def\bX{\boldsymbol X}
\def\bY{\boldsymbol Y}
\def\bbeta{\boldsymbol \beta}
\def\blambda{\boldsymbol \lambda}
\def\bepsilon{\boldsymbol \epsilon}
\def\bone{\boldsymbol{1}}
\def\bzero{\boldsymbol 0}
\def\E{\mbox{E}}
\def\var{\mbox{var}}
\def\gauss{\mbox{N}}
\def\lap{\mbox{L}}
\def\G{\mbox{G}}
\def\go{  $\R$  ightarrow}
\def\invG{\mbox{G}^{-1}}
\def\argmin{\arg\min}

\newtheorem{theorem}{Theorem}
\newtheorem{proposition}{Proposition}
\newtheorem{lemma}{Lemma}
\newtheorem{corollary}{Corollary}
\newtheorem{algorithm}{Algorithm}
\newtheorem{definition}{Definition}
\newtheorem{remark}{Remark}

\usepackage{color}
\usepackage[dvipsnames]{xcolor}
\definecolor{codegreen}{rgb}{0,0.6,0}
\definecolor{codegray}{rgb}{0.5,0.5,0.5}
\definecolor{codepurple}{rgb}{0.58,0,0.82}
\definecolor{backcolour}{rgb}{0.95,0.95,0.92}
\lstdefinestyle{displaycode}{
	backgroundcolor=\color{backcolour},   
	commentstyle=\color{codegreen},
	keywordstyle=\color{magenta},
	numberstyle=\tiny\color{codegray},
	stringstyle=\color{codepurple},
	basicstyle=\footnotesize,
	breakatwhitespace=false,         
	breaklines=true,                 
	captionpos=b,                    
	keepspaces=true,                 
	columns=flexible,
	numbers=none,
	numbersep=5pt,                  
	showspaces=false,                
	showstringspaces=false,
	showtabs=false,                  
	tabsize=2%,
	%xleftmargin=2em,
	%xrightmargin=2em,
} 
\lstdefinestyle{displaycode2}{
	backgroundcolor=\color{backcolour},   
	commentstyle=\color{red},
	keywordstyle=\color{magenta},
	numberstyle=\tiny\color{codegray},
	stringstyle=\color{codepurple},
	basicstyle=\footnotesize,
	breakatwhitespace=false,         
	breaklines=true,                 
	captionpos=b,                    
	keepspaces=true,                 
	columns=flexible,
	numbers=none,
	numbersep=5pt,                  
	showspaces=false,                
	showstringspaces=false,
	showtabs=false,                  
	tabsize=2%,
	%xleftmargin=2em,
	%xrightmargin=2em,
} 

\lstset{style=displaycode}
%\newcommand{\displaycodefile}[1]{\lstinputlisting[language=R]{./codeblocks/#1.txt}}

\newcommand{\pr}{\mathbb{P}}
\newcommand{\ep}{\mathbb{E}}
\renewcommand{\var}{\textrm{Var}}

\renewcommand{\thefootnote}{\fnsymbol{footnote}}
\newcommand{\x}{\bm{x}}
\newcommand{\X}{\bm{X}}
\newcommand{\td}{\textrm{d}}

\newcommand{\tblue}[1]{\textcolor{blue}{#1}}
\newcommand{\tred}[1]{\textcolor{red}{#1}}
\newcommand{\tplum}[1]{\textcolor{Plum}{#1}}
\newcommand{\tcyan}[1]{\textcolor{Cyan}{#1}}
\newcommand{\taqua}[1]{\textcolor{Aquamarine}{#1}}

\newcommand{\CAL}[1]{{\cal #1}}

\begin{document}
\title{\Large \bf STATS 406 F15: Lab 09\\Structured Query Language (SQL) basics}
\date{}

\maketitle

\section{Introduction to relational databases}
\begin{itemize}
	\item The advantage of using relational databases:
	\begin{itemize}[label=*]
		\item A very short article: \url{http://www.teach-ict.com/as_as_computing/ocr/H447/F453/3_3_9/database_design/miniweb/pg8.htm}
		\item Data can be maintained and queried by different threads independently.
		\item More efficient queries for objectives that only concerns one subtable.
	\end{itemize}
	\item Relational database files and their management systems:
	\begin{itemize}[label=*]
		\item A .db file is to relational database management systems (SQLite, MySQL, Oracle, etc) as a .pdf file is to PDF readers (Adobe, Foxit, Okular, etc).
		\item There is one file format (.db) and many tools (management systems) you can use to manage the file.
		\item Different management systems have very similar syntaxes on basic operations and return queried data in very similar forms. They differ in efficiency and other aspects, but not much grammatically.
		\item A list of popular management systems: \url{https://en.wikipedia.org/wiki/Comparison_of_relational_database_management_systems}
		\item In fact, the commands and even most code lines you learned in this class will run without any modification under most management systems.
	\end{itemize}
\end{itemize}

\section{Logistics: using SQLite in R}
\begin{itemize}
	\item This lab is about SQL, not R.
	\item Recap: install package RSQLite and connect to a database.
\begin{lstlisting}[style=displaycode, language=R]
## Load the RSQLite package (if necessary, also install it first):
if(!(require(RSQLite))){
			install.packages("RSQLite", dep=TRUE);
}
## Connect to the file
## Make sure the file is under R's working directory
driver = dbDriver("SQLite");
conn = dbConnect(driver, "baseball.db");
\end{lstlisting}
	
\end{itemize}


\section{SQL commands}

\subsection{A few notes before we start:}
\begin{itemize}
	\item All SQL keywords are case insensitive, but when calling SQL in R, capitalizing key words helps to improve readableness (especially observing a lack syntax coloring).
	\item SQL is designed mainly for extracting data from databases, NOT for analyzing them. SQL provides basic summarizing commands and tools, but do not expect too much. You can use R for further analysis.
\end{itemize}

\subsection{Basic SQL commands:}
\begin{itemize}
	\item The very basic form of SQL queries is:
\begin{lstlisting}[style=displaycode, language=SQL]
/* Pseudo-code */
/* Required clauses */
SELECT ColumnNames
FROM TableName
/* Optional clauses */
WHERE Conditions
GROUP BY VariableNames HAVING Conditions
ORDER BY VariableNames
\end{lstlisting}
	In {\bf SELECT}:
	\begin{itemize}[label=*]
		\item {\bf AS:} It can be used anywhere, not only in {\bf SELECT}. Itself can always be omitted.
		\item As a consequence, the variable name should NOT contain space (Why?). If it is the case in the data, you can escape using `var name' or [var name]. But the correct syntax of this fix needs to be double checked under different platforms.
		\item {\bf AS} renames the extracted variables for convenience. It is especially useful when a. they are summarized; or b. (will see later) when they have to come with prefixes like table names.
		\item Aggregate functions. For a list, check, for example, \url{http://www.techonthenet.com/sql_server/functions/index_alpha.php}.
		\item NOTICE that the wording of specific functions may vary under different management systems. For example, SQLite uses ``length()'' instead of ``len()''.
\begin{lstlisting}[style=displaycode, language=SQL]
/* Pseudo-code: SQLite */
SELECT Var1, Sum(Var2) AS SumVar2, Count(Var3) LengthVar3
FROM TableName
\end{lstlisting}
	\end{itemize}
	In {\bf FROM}:
	\begin{itemize}[label=*]
		\item In this course, unless we combine tables, otherwise we only select from one table.
	\end{itemize}
	In {\bf WHERE}:
	\begin{itemize}[label=*]
		\item If there are multiple conditions, they should be connected by logical connectives ({\bf AND}, {\bf OR} and parenthesis when needed). Conditions can also be decorated by other logical operators ({\bf NOT}, {\bf ANY}, etc). For a list of logical operators in SQL, see \url{http://www.w3resource.com/sql/boolean-operator/sql-boolean-operators.php}.
\begin{lstlisting}[style=displaycode, language=SQL]
/* Pseudo-code: SQLite */
/* Example from: http://beginner-sql-tutorial.com/sql-logical-operators.htm */
SELECT first_name, last_name, age, games 
FROM student_details 
WHERE age >= 10 AND age <= 15 OR NOT games = 'Football'
\end{lstlisting}
	\end{itemize}
	\item For {\bf GROUP BY}:
	\begin{itemize}[label=*]
		\item {\bf GROUP BY} is used in combination with aggregate functions in {\bf SELECT}.
		\item (From Wikipedia) {\bf HAVING} modifies {\bf GROUP BY}. It is indispensable because {\bf WHERE} does not allow aggregate functions.
\begin{lstlisting}[style=displaycode, language=SQL]
/* Pseudo-code: SQLite */
/* Example from: https://en.wikipedia.org/wiki/Having_(SQL) */
SELECT DeptID, Sum(SaleAmount)
FROM Sales
WHERE SaleDate = '01-Jan-2000'
GROUP BY DeptID
HAVING Sum(SaleAmount) > 1000
\end{lstlisting}
	\end{itemize}
\end{itemize}

{\bf Quiz:} Query data from Table \emph{Teams}. Generate a table with the number of teams that won more than half of the games each year. Sort by year.

\begin{lstlisting}[style=displaycode, language=SQL]
# Solution: see Lab_9.r
\end{lstlisting}

\subsection{Inner joining tables}
\begin{itemize}
	\item Basic form (not quite ``basic'', look closely):
\begin{lstlisting}[style=displaycode, language=SQL]
/* Pseudo-code */
SELECT T1.ColumnNames, T2.ColumnNames
FROM TableName1 T1 INNER JOIN TableName2 T2 ON JoiningConditions
/* Other clauses */
WHERE Conditions
/* etc */
\end{lstlisting}
	\item {\bf How does inner join work?}
	\item What is ``TableName1 T1'' doing?
	\item {\bf ON}: Can {\bf ON} be completely replaced by {\bf WHERE}? Within the range of this course, yes, but for aesthetic reasons please don't do so. For more discussion on {\bf ON} vs {\bf WHERE}, see, for example,\\ \url{http://stackoverflow.com/questions/1018822/inner-join-on-vs-where-clause}
	\item {\bf (Optional)} {\bf INNER JOIN} is just the (arguably) the simplest type of joining tables. With other join types, using {\bf ON} or {\bf WHERE} can produce essentially different results. See, for example,\\
	\url{http://blog.sqlauthority.com/2014/10/13/sql-server-what-is-the-difference-between-an-inner-join-and-where-clause/} Start reading from the middle of the page.
	\item In {\bf SELECT} here: in examples you saw in lecture, the variable/column names that {\bf SELECT} picked did not come with prefixes.
	\item \underline{\bf Example:} Read and analyze the following SQL code:
\begin{lstlisting}[style=displaycode, language=SQL]
/* Goal: compare the differences in players' total salaries between teams for each year since 1996. */
SELECT T1.yearID year, T1.teamID Team1ID, T2.teamID Team2ID, T1.SumSalary-T2.SumSalary SalaryDifference
FROM 		(SELECT yearID, teamID, Sum(salary) SumSalary
					FROM Salaries
					GROUP BY yearID, teamID
					ORDER BY yearID, teamID
					) T1
			INNER JOIN
					(SELECT yearID, teamID, Sum(salary) SumSalary
					FROM Salaries
					GROUP BY yearID, teamID
					ORDER BY yearID, teamID
					) T2
			ON
					T1.yearID=T2.yearID AND T1.teamID<T2.teamID
WHERE T1.yearID>1996
GROUP BY T1.yearID, T1.teamID, T2.teamID
ORDER BY T1.yearID, T1.teamID
\end{lstlisting}
	\item NOTICE the trick played with {\bf AS} in {\bf FROM}.
	\item What will happen if we remove the prefix ``T1''/``T2'' from variables in {\bf SELECT}, {\bf GROUP BY} or {\bf ORDER BY}?
	\item What is the effect of feeding {\bf GROUP BY} with two variables?
	\item What is the effect of feeding {\bf ORDER BY} with two variables?
\end{itemize}


\section{Additional resources}
There are many good resources for beginners on SQL on Internet. I just list a few examples:
\begin{itemize}
	\item Stanford online course: \url{https://lagunita.stanford.edu/courses/DB/SQL/SelfPaced/courseware/ch-sql/}
	\item W3School -- \tblue{\bf a good dictionary for beginners}. \url{http://www.w3schools.com/sql/} You can "Try it yourself" but don't expect too high since the example data are sometimes too large to be illustrative.
	\item (Optional) Specifically, if you want to learn more types of join in the future, this is a classical illustration\\ \url{http://stackoverflow.com/questions/6294778/mysql-quick-breakdown-of-the-types-of-joins}.
\end{itemize}


\end{document}